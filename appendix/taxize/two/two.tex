% -*- root: ../../../thesis.tex -*-

\section[Matching species tables]{Matching species tables with different taxonomic resolution} 
\label{ap:taxize:two} 

Trait-based approaches are a promising tool in ecology. Unlike taxonomy-based methods, traits may not be constrained to biogeographic boundaries \citep{baird_toward_2011} and have potential to disentangle the effects of multiple stressors \citep{statzner_can_2010}. 

To analyse trait-composition abundance data must be matched with trait data\-bases like \citep{usseglio-polatera_biological_2000}. However these two datatables may contain species information on different taxonomic levels and perhaps data must be aggregated to a joint taxonomic level.

taxize can help in this data-cleaning step, providing a reproducible workflow. Here we illustrate this on a small fictitious example.

Suppose we have fuzzy coded trait table with 2 traits with 3 respectively 2 modalities:

\begin{knitrout}
\color{fgcolor}\small\begin{kframe}
\begin{alltt}
\hlstd{(traits} \hlkwb{<-} \hlkwd{read.table}\hlstd{(}\hlkwc{header} \hlstd{=} \hlnum{TRUE}\hlstd{,} \hlkwc{sep} \hlstd{=} \hlstr{';'}\hlstd{,} \hlkwc{stringsAsFactors}\hlstd{=}\hlnum{FALSE}\hlstd{,}
                      \hlkwc{text} \hlstd{=} \hlstr{'taxon;T1M1;T1M2;T1M3;T2M1;T2M2
Gammarus sp.;0;0;3;1;3
Potamopyrgus antipodarum;1;0;3;1;3
Coenagrion sp.;3;0;1;3;1
Enallagma cyathigerum;0;3;1;0;3
Erythromma sp.;0;0;3;3;1
Baetis sp.;0;0;0;0;0
'}\hlstd{))}
\end{alltt}
\begin{verbatim}
                     taxon T1M1 T1M2 T1M3 T2M1 T2M2
1             Gammarus sp.    0    0    3    1    3
2 Potamopyrgus antipodarum    1    0    3    1    3
3           Coenagrion sp.    3    0    1    3    1
4    Enallagma cyathigerum    0    3    1    0    3
5           Erythromma sp.    0    0    3    3    1
6               Baetis sp.    0    0    0    0    0
\end{verbatim}
\end{kframe}
\end{knitrout}


And want to match this to a table with abundances:
\begin{knitrout}
\color{fgcolor}\small\begin{kframe}
\begin{alltt}
\hlstd{(abundances} \hlkwb{<-} \hlkwd{read.table}\hlstd{(}\hlkwc{header} \hlstd{=} \hlnum{TRUE}\hlstd{,} \hlkwc{sep} \hlstd{=} \hlstr{';'}\hlstd{,} \hlkwc{stringsAsFactors}\hlstd{=}\hlnum{FALSE}\hlstd{,}
                          \hlkwc{text} \hlstd{=} \hlstr{'taxon;abundance;sample
Gammarus roeseli;5;1
Gammarus roeseli;6;2
Gammarus tigrinus;7;1
Gammarus tigrinus;8;2
Coenagrionidae;10;1
Coenagrionidae;6;2
Potamopyrgus antipodarum;10;1
xxxxx;10;2
'}\hlstd{))}
\end{alltt}
\begin{verbatim}
                     taxon abundance sample
1         Gammarus roeseli         5      1
2         Gammarus roeseli         6      2
3        Gammarus tigrinus         7      1
4        Gammarus tigrinus         8      2
5           Coenagrionidae        10      1
6           Coenagrionidae         6      2
7 Potamopyrgus antipodarum        10      1
8                    xxxxx        10      2
\end{verbatim}
\end{kframe}
\end{knitrout}



First we do some basic data-cleaning and create a lookup-table, that will link taxa in trait table with the abundance table.
\begin{knitrout}
\color{fgcolor}\small\begin{kframe}
\begin{alltt}
\hlcom{# first we remove ' sp.' from out trait table:}
\hlstd{traits}\hlopt{$}\hlstd{taxon_cleaned} \hlkwb{<-} \hlkwd{tolower}\hlstd{(}\hlkwd{gsub}\hlstd{(}\hlstr{" sp."}\hlstd{,} \hlstr{""}\hlstd{, traits}\hlopt{$}\hlstd{taxon))}
\hlcom{# since abundance tables can be very long with repeating taxa, we look only}
\hlcom{# at unique taxon names This will be a lookup-table linking taxon names}
\hlcom{# between both tables}
\hlstd{lookup} \hlkwb{<-} \hlkwd{data.frame}\hlstd{(}\hlkwc{taxon} \hlstd{=} \hlkwd{tolower}\hlstd{(}\hlkwd{unique}\hlstd{(abundances}\hlopt{$}\hlstd{taxon)),} 
            \hlkwc{stringsAsFactors} \hlstd{=} \hlnum{FALSE}\hlstd{)}
\end{alltt}
\end{kframe}
\end{knitrout}


The we query the taxonomic hierarchy for both tables, this will be the backbone of this procedure:
\begin{knitrout}
\color{fgcolor}\small\begin{kframe}
\begin{alltt}
\hlkwd{library}\hlstd{(taxize)}
\hlstd{traits_classi} \hlkwb{<-} \hlkwd{classification}\hlstd{(}\hlkwd{get_uid}\hlstd{(traits}\hlopt{$}\hlstd{taxon_cleaned))}
\hlstd{lookup_classi} \hlkwb{<-} \hlkwd{classification}\hlstd{(}\hlkwd{get_uid}\hlstd{(lookup}\hlopt{$}\hlstd{taxon))}
\end{alltt}
\end{kframe}
\end{knitrout}


First we look if we can find any direct matches between taxon names:
\begin{knitrout}
\color{fgcolor}\small\begin{kframe}
\begin{alltt}
\hlcom{# first search for direct matches}
\hlstd{direct} \hlkwb{<-} \hlkwd{match}\hlstd{(lookup}\hlopt{$}\hlstd{taxon, traits}\hlopt{$}\hlstd{taxon_cleaned)}
\hlcom{# and add the matched name to our lookup table}
\hlstd{lookup}\hlopt{$}\hlstd{traits} \hlkwb{<-} \hlkwd{tolower}\hlstd{(traits}\hlopt{$}\hlstd{taxon[direct])}
\hlstd{lookup}\hlopt{$}\hlstd{match} \hlkwb{<-} \hlkwd{ifelse}\hlstd{(}\hlopt{!}\hlkwd{is.na}\hlstd{(direct),} \hlstr{"direct"}\hlstd{,} \hlnum{NA}\hlstd{)}
\hlstd{lookup}
\end{alltt}
\begin{verbatim}
                     taxon                   traits  match
1         gammarus roeseli                     <NA>   <NA>
2        gammarus tigrinus                     <NA>   <NA>
3           coenagrionidae                     <NA>   <NA>
4 potamopyrgus antipodarum potamopyrgus antipodarum direct
5                    xxxxx                     <NA>   <NA>
\end{verbatim}
\end{kframe}
\end{knitrout}


We found a direct match - \emph{potamopyrgus antipodarum} - so nothing to do here.


Next we look for species which are on a higher taxonomic resolution than our trait table. 
For these species we will take directly the trait-data since no better information is available.

\begin{knitrout}
\color{fgcolor}\small\begin{kframe}
\begin{alltt}
\hlcom{# look for cases where taxonomic resolution in abundance data is higher}
\hlcom{# than in trait data: here we take the trait-values for the lower}
\hlcom{# resolution}.
\hlkwa{for} \hlstd{(i} \hlkwa{in} \hlkwd{which}\hlstd{(}\hlkwd{is.na}\hlstd{(lookup}\hlopt{$}\hlstd{traits))) \{}
    \hlkwa{if} \hlstd{(}\hlkwd{is.data.frame}\hlstd{(lookup_classi[[i]])) \{}
        \hlstd{matches} \hlkwb{<-} \hlkwd{tolower}\hlstd{(lookup_classi[[i]]}\hlopt{$}\hlstd{ScientificName)} \hlopt 
        \hlstd{traits}\hlopt{$}\hlstd{taxon_cleaned}
        \hlkwa{if} \hlstd{(}\hlkwd{any}\hlstd{(matches)) \{}
            \hlstd{lookup}\hlopt{$}\hlstd{traits[i]} \hlkwb{<-} \hlkwd{tolower}
            \hlstd{(lookup_classi[[i]]}\hlopt{$}\hlstd{ScientificName[matches])}
            \hlstd{lookup}\hlopt{$}\hlstd{match[i]} \hlkwb{<-} \hlstd{lookup_classi[[i]]}\hlopt{$}\hlstd{Rank[matches]}
        \hlstd{\}}
    \hlstd{\}}
\hlstd{\}}
\hlstd{lookup}
\end{alltt}
\begin{verbatim}
                     taxon                   traits  match
1         gammarus roeseli                 gammarus  genus
2        gammarus tigrinus                 gammarus  genus
3           coenagrionidae                     <NA>   <NA>
4 potamopyrgus antipodarum potamopyrgus antipodarum direct
5                    xxxxx                     <NA>   <NA>
\end{verbatim}
\end{kframe}
\end{knitrout}


So our abundance data has two \emph{Gammarus} species, however trait data is only on genus level.


The next step is to search for species were we have to aggregate trait-data, since our abundance data is on a lower taxonomic level.
We are walking the taxonomic latter for the species in our trait-data upwards and search for matches with out abundance data. Since we'll have many taxa in the trait-data belonging to one taxon, we'll take the median modality scores as an approximation. Of course also other methods may be used here, e.g. weighting by genetic distance.


\begin{knitrout}
\color{fgcolor}\small\begin{kframe}
\begin{alltt}
\hlcom{# look for cases taxonomic resolution in abundance data is lower than in}
\hlcom{# trait data, here we need to aggregate the trait-values (eg. median value}
\hlcom{# for modality)}
\hlkwa{for} \hlstd{(i} \hlkwa{in} \hlkwd{which}\hlstd{(}\hlkwd{is.na}\hlstd{(lookup}\hlopt{$}\hlstd{traits))) \{}
    \hlcom{# find matches}
    \hlstd{agg} \hlkwb{<-} \hlkwd{sapply}\hlstd{(traits_classi,} \hlkwa{function}\hlstd{(}\hlkwc{x}\hlstd{)} \hlkwd{any}\hlstd{(}
    \hlkwd{tolower}\hlstd{(x}\hlopt{$}\hlstd{ScientificName)} \hlopt
        \hlstd{lookup}\hlopt{$}\hlstd{taxon[i]))}
    \hlkwa{if} \hlstd{(}\hlkwd{sum}\hlstd{(agg)} \hlopt{>} \hlnum{1}\hlstd{) \{}
        \hlcom{# add taxon as aggregate to trait-table}
        \hlstd{traits} \hlkwb{<-} \hlkwd{rbind}\hlstd{(traits,} \hlkwd{c}\hlstd{(}\hlkwd{paste0}\hlstd{(lookup}\hlopt{$}\hlstd{taxon[i],} \hlstr{"-aggregated"}\hlstd{),} 
        \hlkwd{apply}\hlstd{(traits[agg,}
            \hlnum{2}\hlopt{:}\hlnum{6}\hlstd{],} \hlnum{2}\hlstd{, median),} \hlkwd{paste0}\hlstd{(lookup}\hlopt{$}\hlstd{taxon[i],} \hlstr{"-aggregated"}\hlstd{)))}
        \hlcom{# fill lookup table}
        \hlstd{lookup}\hlopt{$}\hlstd{traits[i]} \hlkwb{<-} \hlkwd{paste0}\hlstd{(lookup}\hlopt{$}\hlstd{taxon[i],} \hlstr{"-aggregated"}\hlstd{)}
        \hlstd{lookup}\hlopt{$}\hlstd{match[i]} \hlkwb{<-} \hlstr{"aggregated"}
    \hlstd{\}}
\hlstd{\}}
\hlstd{lookup}
\end{alltt}
\begin{verbatim}
#                      taxon                    traits      match
# 1         gammarus roeseli                  gammarus      genus
# 2        gammarus tigrinus                  gammarus      genus
# 3           coenagrionidae coenagrionidae-aggregated aggregated
# 4 potamopyrgus antipodarum  potamopyrgus antipodarum     direct
# 5                    xxxxx                      <NA>       <NA>
\end{verbatim}
\end{kframe}
\end{knitrout}


Finally we have only one taxon left - clearly an error. We remove this from our dataset:
\begin{knitrout}
\color{fgcolor}\small\begin{kframe}
\begin{alltt}
\hlstd{abundances} \hlkwb{<-} \hlstd{abundances[}\hlopt{!}\hlstd{abundances}\hlopt{$}\hlstd{taxon} \hlopt{==} \hlstd{lookup}\hlopt{$}\hlstd{taxon[}\hlkwd{is.na}\hlstd{(}
                        \hlstd{lookup}\hlopt{$}\hlstd{traits)],}
    \hlstd{]}
\end{alltt}
\end{kframe}
\end{knitrout}



Now we can create \emph{species x sites} and \emph{traits x species} matrices, which could be plugged into methods to analyse trait responses [28].


\begin{knitrout}
\color{fgcolor}\small\begin{kframe}
\begin{alltt}
\hlcom{# species (as matched with trait table) by site matrix}
\hlstd{abundances}\hlopt{$}\hlstd{traits_taxa} \hlkwb{<-} \hlstd{lookup}\hlopt{$}\hlstd{traits[}\hlkwd{match}\hlstd{(}\hlkwd{tolower}\hlstd{(abundances}\hlopt{$}\hlstd{taxon), 
                lookup}\hlopt{$}\hlstd{taxon)]}
\hlkwd{library}\hlstd{(reshape2)}
\hlcom{# reshape data to long format and name rows by samples}
\hlstd{L} \hlkwb{<-} \hlkwd{dcast}\hlstd{(abundances, sample} \hlopt{~} \hlstd{traits_taxa,} \hlkwc{fun.aggregate} \hlstd{= sum,} 
          \hlkwc{value.var} \hlstd{=} \hlstr{"abundance"}\hlstd{)}
\hlkwd{rownames}\hlstd{(L)} \hlkwb{<-} \hlstd{L}\hlopt{$}\hlstd{sample}
\hlstd{L}\hlopt{$}\hlstd{sample} \hlkwb{<-} \hlkwa{NULL}
\hlstd{L}
\end{alltt}
\begin{verbatim}
#   coenagrionidae-aggregated gammarus potamopyrgus antipodarum
# 1                        10       12                       10
# 2                         6       14                        0
\end{verbatim}
\begin{alltt}
\hlcom{# traits by species matrix}
\hlstd{Q} \hlkwb{<-} \hlstd{traits[,} \hlnum{2}\hlopt{:}\hlnum{7}\hlstd{][}\hlkwd{match}\hlstd{(}\hlkwd{names}\hlstd{(L), traits}\hlopt{$}\hlstd{taxon_cleaned), ]}
\hlkwd{rownames}\hlstd{(Q)} \hlkwb{<-} \hlstd{Q}\hlopt{$}\hlstd{taxon_cleaned}
\hlstd{Q}\hlopt{$}\hlstd{taxon_cleaned} \hlkwb{<-} \hlkwa{NULL}
\hlstd{Q}
\end{alltt}
\begin{verbatim}
#                           T1M1 T1M2 T1M3 T2M1 T2M2
# coenagrionidae-aggregated    0    0    1    3    1
# gammarus                     0    0    3    1    3
# potamopyrgus antipodarum     1    0    3    1    3
\end{verbatim}
\begin{alltt}
\hlcom{# check}
\hlkwd{all}\hlstd{(}\hlkwd{rownames}\hlstd{(Q)} \hlopt{==} \hlkwd{colnames}\hlstd{(L))}
\end{alltt}
\begin{verbatim}
# [1] TRUE
\end{verbatim}
\end{kframe}
\end{knitrout}


This is just an example how taxonomic APIs (via taxize) could be used to search for matches up- and downwards the taxonomic ladder. We are looking forward to integrate other databases into taxize, which will facilitate trait-based analyses in R.


%% ----------------------------------------------------------------------------
\section*{References}
\printbibliography[heading=none]

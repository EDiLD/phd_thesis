% -*- root: ../../thesis.tex -*-

\chapter{Introduction and Objectives}
\addthumb{\thechapter}{\Huge\thechapter}{white}{gray}
\label{sec:introduction} 

%% ----------------------------------------------------------------------------
\section{Threats to freshwater ecosystems from chemical pollution}

Freshwater ecosystems, like streams, lakes and wetlands, make up only 0.01\% of the World's water and cover only 0.8\% of Earth's surface \citep{dudgeon_freshwater_2006}, yet they host an important component of global biodiversity. 
Freshwaters are a habitat for more than 125,000 species, which represents 10\% of global biodiversity and \sfrac{1}{3} of all vertebrate species \citep{balian_freshwater_2007,  strayer_freshwater_2010} and provide essential services for human well-being \citep{aylward_freshwater_2005}. 
Small water bodies are of particular importance, because of their high abundance \citep{downing_global_2012}, the high biodiversity they host \citep{davies_comparative_2008} and the ecosystem services they provide \citep{biggs_importance_2016}. 

The earth is currently experiencing a functional change driven by human activities which are so far-reaching, that a new geological epoch "Anthropocene" has been proposed \citep{steffen_2011, waters_anthropocene_2016}. 
Consequently, this is also associated with detrimental biotic changes: 65\% of rivers are currently at threat \citep{vorosmarty_global_2010}, 21\% of 27,516 assessed freshwater species are currently threatened with extinction \citep{iucn_iucn_2016} and freshwaters are the ecosystem experiencing the greatest losses of biodiversity \citep{wwf_living_2016}. 
A multitude of stressors contribute to this deterioration of freshwater biodiversity including habitat loss and degradation, overexploitation, invasive species and pollution \citep{dudgeon_freshwater_2006, vorosmarty_global_2010, wwf_living_2016}. 
Studies investigating water pollution have mainly focused on nutrient loading, acidification and pollution by organic loading \citep{schafer_contribution_2016}. 
However, chemicals have become ubiquitous throughout humankind. 
Currently, more than 100,000 chemicals are registered and in daily use \citep{schwarzman_new_2009, schwarzenbach_global_2010}. 
These substances will ultimately end somewhere in the environment.

Despite their potential negative effects on biota and humans and their intentional release, pesticides have been neglected in the past by ecological studies investigating threats to freshwaters \citep{schafer_contribution_2016} and it is unknown how much they contribute to biodiversity loss \citep{rockstrom_safe_2009, persson_confronting_2013}. 
However, recent studies indicated that pollution by pesticides may be a frequent threat to freshwaters that might have been neglected by ecological studies in the past.
\citet{malaj_organic_2014} showed that almost half of European water bodies are at risk from pesticides. 
In the United States, \citet{stone_pesticides_2014} showed that 61\% of assessed agricultural streams exceed aquatic-life benchmarks.
On a global scale, \citet{stehle_pesticide_2015} found that 52.4\% of detected insecticide concentrations (n~=~11,300) exceeded risk thresholds.
The high contact with adjacent land and low water volume of small streams make them particularly vulnerable to pesticide pollution \citep{biggs_importance_2016}, however, there is currently a lack of data on pesticide pollution of small streams \citep{lorenz_specifics_2016}. 

As a reaction to the degradation of freshwaters, several legal frameworks have been established to safeguard and improve the quality of freshwater ecosystems. 
In the European Union (EU), the Water Framework Directive (WFD) \citep{european_union_directive_2000} regulates the protection of aquatic ecosystems and commits the member states to achieve a `good' status of all water bodies. 
Knowing of the toxicity of pesticides and their intentional release into the environment, also the introduction and use of new pesticides are highly regulated.
Sophisticated environmental risk assessment procedures have been developed and are requested by the EU \citep{european_union_regulation_2009} to ensure that the use of pesticides does not cause unacceptable effects to non-target organism, soil, air and water. 



%% ----------------------------------------------------------------------------
\section{Environmental Risk Assessment}

Environmental risk assessment (ERA) tries to estimate risks to animals, populations or ecosystems.
It investigates if a chemical can be used as intended without causing detrimental impacts to the environment. 
Moreover, ERA is used as a tool to support decision making under uncertainty \citep{newman_fundamentals_2015}. 
Environmental risk is defined as a combination of the severity and the probability of occurrence of a potential adverse effect on the environment \citep{suter_ecological_2007}. 
Therefore, ERA is based on two components: Effect- and exposure assessment.
A combination of both is needed to characterise environmental risks.

% Effect assessment
Effect assessment characterises the strength of effects using laboratory and semi-field experiments.
It establishes relationships between the concentration of a compound and the observed effects.
In the European Union a tiered approach with increasing complexity and realism.
Lower tier assessment is based on highly standardised single species laboratory experiments, whereas higher tier assessment is refined by testing additional species, extended laboratory experiments or model ecosystem experiments \citep{brock_aquatic_2006}. 
To address the various uncertainties in effect assessment (e.g. experimental variation, variation between species, variation in environmental conditions etc.) the retrieved toxicity values are divided by an assessment factor (AF) between 100 (lower tier assessment) and 2 (higher tier assessment) depending on data quality, which yields to a regulatory acceptable concentration (RAC) \citep{efsa_guidance_2013, brock_aquatic_2006}. 

% Exposure assessment
Exposure Assessment for freshwaters aims to characterise the probability of an adverse effect by deriving a predicted environmental concentration (PEC) in surface waters and sediments \citep{newman_fundamentals_2015}. 
It is mainly based on modelling the fate of chemicals in the environment using computer simulations. 
In the European Union, the FOCUS models are used \citep{focus_focus_2001, efsa_guidance_2013}.
To calculate PECs these models need many compound specific input parameters like the molecular weight, water solubility, partitioning coefficients and dissipation time. 
Additionally, information on the application regime and crop type is needed. 
FOCUS models the concentration within edge-of-field streams of 1~meter width (corresponding a catchment size of approx. 7km\textsuperscript{2}, see Figure~\ref{fig:size_width}) and 30~cm depth \citep{erlacher_regulation_2011}. 
Nevertheless, recent research showed that FOCUS models fail to predict measured field concentrations of pesticides \citep{knabel_regulatory_2012, knabel_fungicide_2014}. 

% Risk characterisation
The final step in ERA is risk characterisation.
It puts together the information gained from effect and exposure assessment. 
Risk can be expressed in several ways, a quantitative way being the risk quotient approach: If the ratio PEC / RAC exceeds a value of one potential risks cannot be rebutted \citep{efsa_guidance_2013, suter_ecological_2007,  solomon_probabilistic_2000}. 
Consequently, pesticides can be authorised only if the risk quotient is below one indicating that harmful effects are unlikely.



%% ----------------------------------------------------------------------------
\section{Environmental Monitoring}

Widespread anthropogenic activities and the induced environmental changes have resulted in concerns about the state of the environment and have led to the development of environmental monitoring programs worldwide \citep{nichols_monitoring_2006}. 
After authorization, pesticides applied on agricultural fields may enter aquatic ecosystems via diffuse sources like spray-drift, surface run-off or drainage \citep{schulz_field_2004, stehle_probabilistic_2013, liess_determination_1999, carter_how_2000}. 
These entered pesticides may have ecological effects and worsen the chemical status, acting contrary to the goal of the WFD. 
For monitoring the progress towards the goal of a `good' status and for assessment of the chemical status of surface waters the EU WFD established monitoring requirements for all European river basins \citep{european_union_directive_2000}. 
For chemical monitoring the WFD requires grab sampling and chemical analysis of 21 priority substances (of which 7 are pesticides) every third month and of 24 other pollutants (of which 12 are used as pesticides) every month and derived for these environmental quality standards (EQS) \citep{european_union_directive_2013}. 
Additionally, 14 substances (of which 8 are used as pesticides, including all Neonicotinoids) that may pose a significant risk, have an insufficient data basis and are candidates for future priority substances are currently monitored until 2019 \citep{european_union_commission_2015}.
Nevertheless, monitoring programs on a national scale might monitor a broader spectrum of chemical substances, e.g. for investigative monitoring. 
Recent studies indicate that the current sampling and chemical analyses strategy greatly underestimate the pesticide exposure \citep{stehle_probabilistic_2013, xing_influences_2013, moschet_how_2014}. 

% Monitoring and ERA
Environmental monitoring produces humongous amounts of data containing information on pesticide concentrations in the field on a large under many conditions.
Therefore, it can be complementary to ERA \citep{suter_ecological_2007}. 
Moreover, data from long-term monitoring programs can be used to study hypotheses about spatial and temporal dynamics and interactions, that are not evident from short term and short scale studies \citep{gitzen_design_2012} and provide insights modelling approaches. 
If the environmental risk assessment process captured all relevant sources of risk, no concentrations above the derived RAC should be observable in European rivers. 
Therefore, monitoring data could be used to provide feedback for ERA after approval \citep{knauer_pesticides_2016}. 
The WFD has its main focus on large water bodies \textgreater 10~$km^2$ catchmentsize \citep{european_union_directive_2000}, whereas ERA has its focus on small water bodies \textless 10~$km^2$ \citep{european_union_regulation_2009, brock_aquatic_2006}.
However, at present little is known on pesticide concentrations in small streams comparable to those assessed in ERA \citep{lorenz_specifics_2016, biggs_importance_2016}. 



%% ----------------------------------------------------------------------------
\section{Statistical Ecotoxicology}

Environmental effect assessment generates data on ecological effects using experiments. 
The produced datasets range from small univariate datasets (lower tier assessment) to medium sized multivariate datasets (higher tier assessment).
% Statistical Ecotoxicology
In order to extract usable information for assessment, these datasets are analysed using statistical techniques and therefore, statistics are crucial for effect assessment \citep{newman_quantitative_2012}.
Statistical ecotoxicology combines statistics with the specific needs and constraints of ecotoxicology. 
Ecotoxicologists deal generally with low replicated experiments, making statistical inference difficult \citep{van_der_hoeven_power_1998}.
For example, a recent analysis of eleven mesocosm studies revealed that the sample sizes for these kind of experiments range between two and five.
Statistical ecotoxicology aims to provide solutions to statistical challenges in ecotoxicology \citep{fox_comment_2016}, guidance on experimental designs \citep{johnson_power_2015} and tools to integrate big data \citep {van_den_brink_new_2016}.
The ultimate goal is to improve the accuracy of ERA. 

% Dose-response, NOEC
The relationships between the concentration of a compound and the observed effects are usually analysed using dose-response models, which can be used to derive an effective concentration for x\% effect ($EC_{x}$) \citep{ritz_toward_2010}. 
Nevertheless, such relationships cannot always be established from experimental data.
For example, mesocosm experiments are conducted to characterise effects on whole biological communities.
However, because of multivariate responses and potential indirect effects, there is no clear dose-response relationship and no models for this kind of data available. 
There are also examples were fitting dose-response models is problematic \citep{green_issues_2016}. 
In such cases, there is usually a no-observed-effect concentration (NOEC) computed. 

The NOEC is the highest tested concentration that does not lead to significant deviation from the control response and therefore relies on null hypothesis significance testing (NHST). 
However, the use of NOEC as a toxicity measure in environmental effect assessment has been heavily criticised in the past \citep{laskowski_good_1995, chapman_warning:_1996, warne_noec_2008, fox_what_2012, jager_bad_2012, fox_dont_2016}. 
One such critic is the low statistical power for NHST in common ecotoxicological experiments \citep{van_der_hoeven_power_1998}.
\emph{A priori} power calculations can provide useful guidance for choosing experimental designs \citep{johnson_power_2015}, but are rarely used by ecotoxicologists \citep{newman_what_2008}. 

Instead of conducting experiments, toxicity could be also predicted from molecular structures using quantitative structure-activity relationships (QSAR), which are usually calculated using machine-learning techniques \citep{murrell_chemically_2015, cortes-ciriano_bioalerts:_2016}. 
Nevertheless, in order to improve and validate these models to give sufficient prediction accuracy more data from experiments is needed \citep{kuhne_read-across_2013}. 
% Big data
Indeed, a large amount of data is available that could be used for effect and exposure assessment. 
For example, the US EPA ECOTOX database \citep{u.s._epa_ecotoxicology_2016}, the Pesticides Properties Database \citep{lewis_international_2016} and ETOX \citep{umweltbundesamt_etox:_2016} provide toxicity data that could be used for effect assessment.
Databases like Physprop \citep{howard_physical_2016} and PubChem \citep{kim_pubchem_2016} provide chemical properties that are needed as input for exposure models.
Monitoring data provides information on realised concentrations, could be used for validation of models and retrospective risk assessment.
This "big data" can provide new information and opportunities for ERA \citep{dafforn_big_2015}. 
However, it needs to be harmonised, linked and easily accessible in order to be used effectively in ERA.



%% ----------------------------------------------------------------------------
\newpage
\section{Objectives and Outline of the thesis}

The overall goal of this thesis was to contribute to the emerging field of statistical ecotoxicology, environmental risk assessment and environmental monitoring.
The main objectives were (i) to scrutinise new methods in statistical ecotoxicology,
(ii) explore available monitoring data and
(iii) provide tools to deal with big data.
Figure \ref{fig:intro:overview} provides a conceptual overview on ERA and environmental monitoring as outlined in the previous sections, as well as the parts considered in this thesis and the relations between them. 

\begin{figure}[h]
	\vspace{2em}
    \hspace*{-1cm} 
	\resizebox{1.1\textwidth}{!}{%
		% -*- root: ../../../thesis.tex -*-

% \tikzsetnextfilename{overview}
\usetikzlibrary{shapes, arrows, positioning, calc, arrows.meta}

% Define elements
% arrows, see also http://tex.stackexchange.com/questions/5461/is-it-possible-to-change-the-size-of-an-arrowhead-in-tikz-pgf/161238#161238
\tikzstyle{line} = [draw, -{Latex[length=4mm,width=3mm]}, ultra thick]

\tikzstyle{block} = [rectangle, draw, 
    text width=5em, text centered, rounded corners, minimum height=4em]

% papers
\tikzstyle{paper} = [circle, draw, fill=gray!85, fill opacity=0.4, text opacity=1,  font = \bf\Large, minimum width=2.5cm]
\tikzstyle{textbf} = [text centered, font = \bf\Large]

\begin{tikzpicture}[node distance = 2cm, auto]
	\stopthumb
	
% clip figure
\clip(-1.5,-10.5) rectangle (22.2,5.2);

% % % % grid for coordinates for clip
% \draw[help lines,xstep=1,ystep=1] (-2,-13) grid (30,6.5);
% \foreach \x in {-2,-1,...,30} { \node [anchor=north] at (\x,0) {\x}; }
% \foreach \y in {-13,-12,...,6} { \node [anchor=east] at (0,\y) {\y}; }


% Nodes
	%% Effects
	\node [name = exp, block, minimum width=2cm] {Experiment} ;
	\node [name = stat, block, minimum width=2cm, right=1cm of exp] {Data / Statistics} ;
    \node [name = eff, block, 
		minimum width=57mm, 
		minimum height=25mm, 
	below left=5mm of exp.west, anchor = west] {} ;
	\node[textbf, below right=8mm and 5mm of exp, anchor = south]{Effects};

	%% Exposure
  	\node [name = prop, block, minimum width=2cm, below=38mm of exp] {Data / Properties} ;
	\node [name = model, block, minimum width=2cm, right=1cm of prop] {Models} ;
	\node [name = expo, block, 
		minimum width=57mm, 
		minimum height=25mm, 
		below = 20mm of eff] {} ;
	\node[textbf, above=-2mm of expo, anchor = north]{Exposure};

	%% Risk Assessment
	\node [name = risk, block, below right=0.75cm and 1cm of stat,
        minimum width=45mm, 
		minimum height=2.5cm, 
		font = \bf\large,
		align = center,
       text width = 3cm] {Environmental Risk\\  Assessment};

	%% Monitoring data
	\node [name = monit, block, 
		right = 8cm of risk,
        minimum width=45mm, 
		minimum height=25mm, 
		font = \bf\large,
		align = center,
       text width = 3cm] {Environmental\\ Monitoring};

	%% biological data
	\node [name = bio, block, 
		above left = 2cm and 2cm of monit, anchor = north,
		minimum width=30mm, align = center, text width = 30mm, font=\large] {Biology   };
	%% chemical data
	\node [name = chem, block, 
		below left = 2cm and 2cm of monit, anchor = south,
		minimum width=3cm, font=\large,text width = 30mm] { Chemistry};


  %% Chapters
	\node[name = chap2, paper, 
		above left = 9mm and -15mm of stat, 
		anchor = east]{Chap. 2};	
    \node[name = chap3, paper, 
		below left = -32mm and 3mm of chem, anchor = north,
		]{Chap. 3};
	\node[name = chap4, paper, anchor = north, yshift=-25mm,  xshift = 10mm,
		] (chap4) at ($(chem)!0.5!(expo)$) {Chap. 4};
	\node[name = chap5, paper, anchor = south, yshift=25mm, xshift = 10mm,
		] (chap5) at ($(bio)!0.5!(eff)$) {Chap. 5};
   \node[name=rl1, above= 0mm of chap5]{Retrieve \& link data};
   \node[name=rl1, below= 0mm of chap4]{Retrieve \& link data};

% arrows
	\path [line] (exp) -- (stat);
	\path [line] (prop) -- (model);
	\path [line] (eff) -| node[pos = 0.4, font = \large]{RAC} (risk);
	\path [line] (expo) -| node[pos = 0.4, font = \large,  below]{PEC} (risk);
	\path [line] (monit) |- (chem);
	\path [line] (monit) |- (bio);
	\path [line, dashed] (chap4) edge [bend left = 20]  (prop);
    \path [line, dashed] (chap5) edge [bend right = 20]   (stat);
	\path [line, dashed] (chap4) edge [bend right = 20]   (chem);
    \path [line, dashed] (chap5) edge [bend left = 20]   (bio);
    \path [dashed] (chap3.north west) edge [line, bend right = 30] node[xshift = 25mm, yshift =5mm, font = \large, align = center, fill = white] {Retrospection}  (risk);
	\path [dashed] (risk.south east) edge [line ,bend right = 40]  node [xshift = 10mm, pos =0.2,  below, font = \large, align = center, fill = white] {Approves \\ Substance} (chem);


\end{tikzpicture}
\continuethumb 

	}
	\caption[Conceptual overview of the topics addressed by this thesis]{Conceptual overview on environmental risk assessment, environmental monitoring and the parts addressed by this thesis.}
	\label{fig:intro:overview}
\end{figure}


\clearpage
\noindent The thesis starts with a comparison of statistical methods to analyse ecotoxicological experiments using NHST in effect assessment (Chapter \ref{sec:usetheglm}). 
Specific questions addressed were:

\begin{itemize}
	\item Are newer statistical methods, explicitly considering the type of analysed data, more powerful than currently used methods for NHST?
	\item How much statistical power do current experimental designs in ecotoxicology exhibit?
\end{itemize}


\noindent Risk assessment procedures in the European Union has it main focus on small waterbodies adjacent to agricultural fields where plant protection products are applied.
Therefore, chapter \ref{sec:smallstreams} focuses on measured large-scale environmental concentrations in small streams, the drivers thereof and comparison with RACs derived from ERA.
Specific goals of this study were:
\begin{itemize}
	\item Compile monitoring data on pesticides in small streams in Germany and check if the available data is suitable to inform ERA.
	\item Explore the relationship between agricultural land use and stream size and RAC exceedances.
	\item Scrutinise the annual dynamics of pesticide exposure, as well as the influence of precipitation on measured pesticide concentrations.
	\item We use RACs derived from ERA to assess the current pollution in German streams and identify pesticides exhibiting currently a risk to freshwaters.
\end{itemize}

\noindent
The compilation of monitoring data from different data sources in Chapter \ref{sec:smallstreams}, resulted in a big inhomogeneous amount of data.
Moreover, Biologists, Chemists and ecotoxicologists face similar problems with the need to identify and harmonise their biological and chemical data.
Chapters \ref{sec:webchem} (chemical data) and \ref{sec:taxize} (biological data) describe software solutions to simplify and accelerate the workflow of:

\begin{itemize}
	\item validating and harmonising chemical and taxonomic data
	\item linking datasets from different databases
	\item retrieving properties and identifiers
\end{itemize}






%% ----------------------------------------------------------------------------
\clearpage
\section{References}
\printbibliography[heading=none]

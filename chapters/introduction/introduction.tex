\chapter{Introduction and Objectives}
\label{sec:introduction} 

\section{Statistical Ecotoxicology}

\section{Chemical pollution of freshwater ecosystems}

\section{Dealing with \emph{big data} in eco(toxico-)logy}




\section{Objectives and structure of the thesis}

This thesis pursues three objectives: 
\begin{enumerate}[i]
	\item to scrutinize new methods in statistical ecotoxicology,
	\item explore available monitoring data for ecological risk assessment and
	\item provide tools for scientists to deal more effective with the data they generate.
\end{enumerate}


The thesis starts with a comparison of statistical methods for ecotoxicological experiments (Chapter \ref{sec:usetheglm}). 
Specific questions addressed were:

\begin{itemize}
	\item Are statistical methods taking the nature of the data into account more powerful than currently used methods?
	\item How much statistical power do current experimental designs ecotoxicology have?
\end{itemize}








Chapters \ref{sec:webchem} (chemical data) and \ref{sec:taxize} (biological data) describe software solutions to simplify and accelerate the workflow of:

\begin{itemize}
	\item validating chemical and taxonomic names
	\item link them to other dataset
	\item search properties and identifiers
\end{itemize}

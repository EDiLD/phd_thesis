% -*- root: ../../thesis.tex -*-

\chapter{Introduction and Objectives}
\label{sec:introduction} 

%% ----------------------------------------------------------------------------
\section{Pesticides in freshwater ecosystems}



%% ----------------------------------------------------------------------------
\section{Ecological Risk Assessment}
%! Link to previous section
Ecological risk assessment (ERA) tries to estimate risks to animals, populations or ecosystems and is used as a tool for decision making under uncertainty \citep{newman_fundamentals_2015}. 
The decision to be made is, whether a (new) pesticide can be approved for usage and a potential release in the environment without a risk to the environment. 
Ecological risk is defined as a combination of the severity and the probability of occurrence of a potential adverse effect \citep{suter_ecological_2007}. 
Therefore, ERA is based on two components: Effect- and exposure assessment.
A combination of both is needed to characterise ecological risks.

% Effect assessment
Effect assessment characterises the strength of effects using laboratory and semi-field experiments.
It establishes relationships between the concentration of a compound and the observed ecological effects.
In the European Union a tiered approach with increasing complexity and realism.
Lower tier assessment is based on highly standardised single species laboratory experiments, whereas higher tier assessment is refined by testing additional species, extended laboratory experiments or model ecosystem experiments. 
To address the various uncertainties in effect assessment (e.g. experimental variation, variation between species, variation in environmental conditions etc) the retrieved toxicity values are multiplied by an assessment factor between 0.01 (lower tier assessment) and 0.5 (higher tier assessment) depending on data quality, which yields to a regulatory acceptable concentration (RAC) \citep{efsa_guidance_2013}. 

% Exposure assessment
Exposure Assessment for freshwaters aims to characterise the probability of an adverse effect by deriving a predicted environmental concentration (PEC) in surface waters and sediments \citep{newman_fundamentals_2015}. 
It is mainly based on modeling the fate of chemicals in the environment using computer simulations. 
In the European Union, the FOCUS models are used \citep{focus_focus_2001, efsa_guidance_2013}.
To calculate PECs these models need many compound specific input parameters like the molecular weight, water solubility, partitioning coefficients and dissipation time. 
Additionally, information on the application regime and crop type is needed. 
FOCUS models the concentration within edge-of-field streams of 1~meter width and 30cm depth \citep{erlacher_regulation_2011}. 
Nevertheless, recent research showed that FOCUS models fail predict measured field concentrations of pesticides \citep{knabel_regulatory_2012, knabel_fungicide_2014}. 

% Risk characterisation
The final step in ERA is risk characterisation.
It puts together the information gained from effect and exposure assessment. 
Risk can be expressed in several ways, a quantitative way being the risk quotient approach: A PEC / RAC ratio greater than one indicating potential risks \citep{efsa_guidance_2013, suter_ecological_2007, amiard-triquet_aquatic_2015}. 
Substances with a ratio lower than one could be approved for usage and potential release to the environment.



%% ----------------------------------------------------------------------------
\section{Environmental Monitoring}

Concerns about the environmental state have lead to extensive monitoring activities. 
Widespread anthropogenic activities induced environmental changes have resulted in concerns about the state of the environment and have lead to the development of environmental monitoring programs worldwide \citep{nichols_monitoring_2006}. 
In Europe, the Water Framework Directive (WFD) \citep{european_union_directive_2000} establishes monitoring requirements for all European river basins. 
These monitoring efforts are used to estimate the environmental state and trends. 

Environmental monitoring can be complementary to ecological risk assessment \citep{suter_ecological_2007}.
Moreover, data from long-term monitoring programs can be used to study hypotheses about spatial and temporal dynamics and interactions, that are not evident from short term and short scale studies \citep{gitzen_design_2012}.
Therefore, monitoring data could be used to inform and review ERA after approval \citep{knauer_pesticides_2016}. 
However, there is a mismatch between streams assessed: The WFD aims at monitoring medium size to large streams greater than 10~$km^2$ catchment size, whereas ERA assesses risks for streams corresponding to a catchment size of approximately 7~$km^2$ (corresponding to 1~meter width, see Figure~\ref{fig:size_width} \textcolor{red}{ref to small streams supplement}). 


%%! Mehr auf UBA projekt eingehen.
%%! Hier den Link zu Stehle reinbringen?
%%! Malaj einbauen!

%% ----------------------------------------------------------------------------
\section{Statistical Ecotoxicology}
% Link to previous section
Ecological effect assessment generates data on ecological effects using experiments. 
The produced datasets range from small univariate datasets (lower tier assessment) to medium sized multivariate datasets (higher tier assessment).
% Statistical Ecotoxicology
These datasets are analysed using statistical techniques in order to extract usable information for assessment and therefore, statistics are crucial for effect assessment \citep{newman_quantitative_2012}.
Statistical ecotoxicology combines statistics with the specific needs and constraints of ecotoxicology. 
It aims to provide solutions to statistical challenges in ecotoxicology \citep{fox_comment_2016}, guidance on experimental designs \citep{johnson_power_2015} and tools to integrate big data \citep {van_den_brink_new_2016} to improve accuracy of ERA. 

% Dose-response, NOEC
The relationships between the concentration of a compound and the observed effects are usually analysed using dose-response models, which can be used to derive an effective concentration for x\% effect ($EC_{x}$) \citep{ritz_toward_2010}. 
Nevertheless, such relationships cannot always be established from experimental data.
For example, model ecosystem experiments are conducted to characterise effects on whole biological communities.
However, because of multivariate responses and potential indirect effects, there is no clear dose-response relationship and no models for this kind of data available. 
There are also other examples were fitting dose-response models is problematic \citep{green_issues_2016}. 
In such cases, there is usually a no-observed-effect concentration (NOEC) computed. 

The NOEC is the highest tested concentration that does not lead to significant deviation from the control response and therefore relies on null hypothesis significance testing (NHST). 
However, the use of NOEC as toxicity measure in ecological effect assessment has been heavily criticised in the past \citep{laskowski_good_1995, chapman_warning:_1996, warne_noec_2008, fox_what_2012, jager_bad_2012, fox_dont_2016}. 
One such critic is the low statistical power for NHST in common ecotoxicological experiments \citep{van_der_hoeven_power_1998}.
\emph{A priori} power calculations can provide useful guidance for choosing experimental designs \citep{johnson_power_2015}, but are rarely used by ecotoxicologists \citep{newman_what_2008}. 

Instead of conducting experiments, toxicity could be also predicted from molecular structures using quantitative structure-activity relationships (QSAR), which are usually calculated using machine-learning techniques \citep{murrell_chemically_2015, cortes-ciriano_bioalerts:_2016}. 
Nevertheless, in order to improve these models to give sufficient prediction accuracy more data from experiments is needed \citep{kuhne_read-across_2013}. 

% Big data
A large amount of data is available that could be used for effect and exposure assessment. 
For example, the US EPA ECOTOX database \citep{u.s._epa_ecotoxicology_2016}, the Pesticides Properties Database \citep{lewis_international_2016} and ETOX \citep{umweltbundesamt_etox:_2016} provide toxicity data that could be used for effect assessment.
Databases like Physprop \citep{howard_physical_2016} and PubChem \citep{kim_pubchem_2016} provide chemical properties that are needed as input for exposure models.
Monitoring data provides information on realised concentrations, could be used for validation of models and retrospective risk assessment.
This "big data" can provide new information and opportunities for ERA \citep{dafforn_big_2015}. 
However, it needs to be linked and easily accessible in order to be used effectively in ERA.



%% ----------------------------------------------------------------------------
\section{Objectives and Outline of the thesis}

The overall goal of this thesis was to contribute to the emerging field of statistical ecotoxicology, ecological risk assessment and environmental monitoring.
The main objectives were (i) to scrutinise new methods in statistical ecotoxicology,
(ii) explore available monitoring data and
(iii) provide tools to deal with big data.
Figure \ref{fig:intro:overview} provides a conceptual overview on ERA and environmental monitoring as outlined in the previous sections, as well as the parts of this thesis and its relations. 

\noindent The thesis starts with a comparison of statistical methods to analyse ecotoxicological experiments in effect assessment (Chapter \ref{sec:usetheglm}). 
Specific questions addressed were:

\begin{itemize}
	\item Are newer statistical methods more powerful than currently used methods for NHST?
	\item How much statistical power do current experimental designs in ecotoxicology exhibit?
\end{itemize}


\noindent Exposure assessment aims at predicting chemical concentrations in small streams. 
Chapter \ref{sec:smallstreams} focuses on measured large-scale environmental concentrations and the drivers thereof. 
Specific goals were:
\begin{itemize}
	\item Compile all available monitoring data on pesticides in small streams in Germany
	\item Explore the relationship between agricultural land use and streams size and measured pesticide concentrations.
	\item Study annual dynamics of pesticide exposure, as well as the influence of precipitation on measured pesticide concentrations.
	\item Assess the current pollution in German streams and identify responsible pesticides.
\end{itemize}

\noindent
The compilation of monitoring data from different data sources, lead to a big inhomogeneous amount of data that first needs to be harmonised.
Chapter \ref{sec:webchem} (chemical data) and Chapter \ref{sec:taxize} (biological data) describe software solutions to simplify and accelerate the workflow of:

\begin{itemize}
	\item validating and harmonising chemical and taxonomic data
	\item linking datasets
	\item retrieving properties and identifiers
\end{itemize}


\begin{figure}
	\resizebox{1\textwidth}{!}{%
		% -*- root: ../../../thesis.tex -*-

% \tikzsetnextfilename{overview}
\usetikzlibrary{shapes, arrows, positioning, calc, arrows.meta}

% Define elements
% arrows, see also http://tex.stackexchange.com/questions/5461/is-it-possible-to-change-the-size-of-an-arrowhead-in-tikz-pgf/161238#161238
\tikzstyle{line} = [draw, -{Latex[length=4mm,width=3mm]}, ultra thick]

\tikzstyle{block} = [rectangle, draw, 
    text width=5em, text centered, rounded corners, minimum height=4em]

% papers
\tikzstyle{paper} = [circle, draw, fill=gray!85, fill opacity=0.4, text opacity=1,  font = \bf\Large, minimum width=2.5cm]
\tikzstyle{textbf} = [text centered, font = \bf\Large]

\begin{tikzpicture}[node distance = 2cm, auto]
	\stopthumb
	
% clip figure
\clip(-1.5,-10.5) rectangle (22.2,5.2);

% % % % grid for coordinates for clip
% \draw[help lines,xstep=1,ystep=1] (-2,-13) grid (30,6.5);
% \foreach \x in {-2,-1,...,30} { \node [anchor=north] at (\x,0) {\x}; }
% \foreach \y in {-13,-12,...,6} { \node [anchor=east] at (0,\y) {\y}; }


% Nodes
	%% Effects
	\node [name = exp, block, minimum width=2cm] {Experiment} ;
	\node [name = stat, block, minimum width=2cm, right=1cm of exp] {Data / Statistics} ;
    \node [name = eff, block, 
		minimum width=57mm, 
		minimum height=25mm, 
	below left=5mm of exp.west, anchor = west] {} ;
	\node[textbf, below right=8mm and 5mm of exp, anchor = south]{Effects};

	%% Exposure
  	\node [name = prop, block, minimum width=2cm, below=38mm of exp] {Data / Properties} ;
	\node [name = model, block, minimum width=2cm, right=1cm of prop] {Models} ;
	\node [name = expo, block, 
		minimum width=57mm, 
		minimum height=25mm, 
		below = 20mm of eff] {} ;
	\node[textbf, above=-2mm of expo, anchor = north]{Exposure};

	%% Risk Assessment
	\node [name = risk, block, below right=0.75cm and 1cm of stat,
        minimum width=45mm, 
		minimum height=2.5cm, 
		font = \bf\large,
		align = center,
       text width = 3cm] {Environmental Risk\\  Assessment};

	%% Monitoring data
	\node [name = monit, block, 
		right = 8cm of risk,
        minimum width=45mm, 
		minimum height=25mm, 
		font = \bf\large,
		align = center,
       text width = 3cm] {Environmental\\ Monitoring};

	%% biological data
	\node [name = bio, block, 
		above left = 2cm and 2cm of monit, anchor = north,
		minimum width=30mm, align = center, text width = 30mm, font=\large] {Biology   };
	%% chemical data
	\node [name = chem, block, 
		below left = 2cm and 2cm of monit, anchor = south,
		minimum width=3cm, font=\large,text width = 30mm] { Chemistry};


  %% Chapters
	\node[name = chap2, paper, 
		above left = 9mm and -15mm of stat, 
		anchor = east]{Chap. 2};	
    \node[name = chap3, paper, 
		below left = -32mm and 3mm of chem, anchor = north,
		]{Chap. 3};
	\node[name = chap4, paper, anchor = north, yshift=-25mm,  xshift = 10mm,
		] (chap4) at ($(chem)!0.5!(expo)$) {Chap. 4};
	\node[name = chap5, paper, anchor = south, yshift=25mm, xshift = 10mm,
		] (chap5) at ($(bio)!0.5!(eff)$) {Chap. 5};
   \node[name=rl1, above= 0mm of chap5]{Retrieve \& link data};
   \node[name=rl1, below= 0mm of chap4]{Retrieve \& link data};

% arrows
	\path [line] (exp) -- (stat);
	\path [line] (prop) -- (model);
	\path [line] (eff) -| node[pos = 0.4, font = \large]{RAC} (risk);
	\path [line] (expo) -| node[pos = 0.4, font = \large,  below]{PEC} (risk);
	\path [line] (monit) |- (chem);
	\path [line] (monit) |- (bio);
	\path [line, dashed] (chap4) edge [bend left = 20]  (prop);
    \path [line, dashed] (chap5) edge [bend right = 20]   (stat);
	\path [line, dashed] (chap4) edge [bend right = 20]   (chem);
    \path [line, dashed] (chap5) edge [bend left = 20]   (bio);
    \path [dashed] (chap3.north west) edge [line, bend right = 30] node[xshift = 25mm, yshift =5mm, font = \large, align = center, fill = white] {Retrospection}  (risk);
	\path [dashed] (risk.south east) edge [line ,bend right = 40]  node [xshift = 10mm, pos =0.2,  below, font = \large, align = center, fill = white] {Approves \\ Substance} (chem);


\end{tikzpicture}
\continuethumb 

	}
	\caption[Conceptual overview of the topics addressed by this thesis]{Conceptual overview on data in ecological risk assessment, environmental monitoring and the parts addressed by this thesis.}
	\label{fig:intro:overview}
\end{figure}





%% ----------------------------------------------------------------------------
\newpage
\section{References}
\printbibliography[heading=none, sorting=nyt]

\chapter[Ecotoxicology is not normal]{Ecotoxicology is not normal - A comparison of statistical approaches for analysis of count and proportion data in ecotoxicology}
\label{sec:usetheglm} 
 
Eduard Szöcs\textsuperscript{a} \& Ralf B. Schäfer\textsuperscript{a} \\

\medskip
\small
\textsuperscript{a}Institute for Environmental Sciences, University Koblenz-Landau, 76829 Landau, Germany \\

\medskip 
\normalsize
Adapted from the article published 2015 in \textbf{Environmental Science and Pollution Research}, 22(18), 13990-13999

\newpage

\section{Abstract}
\label{sec:usetheglm:abstract} 
Ecotoxicologists often encounter count and proportion data that are rarely normally distributed.
To meet the assumptions of the linear model such data are usually transformed or non-parametric methods are used if the transformed data still violate the assumptions.
Generalised Linear Models (GLM) allow to directly model such data, without the need for transformation.
Here, we compare the performance of  two parametric methods, i.e.,  (1) the linear model (assuming normality of transformed data), (2) GLMs (assuming a Poisson, negative binomial, or binomially distibuted response), and (3) non-parametric methods.

We simulated typical data mimicking low replicated ecotoxicological experiments of two common data types (counts and proportions from counts). 
We compared the performance of the different methods in terms of statistical power and Type I error for detecting a general treatment effect and determining the lowest observed effect concentration (LOEC).
In addition, we outlined differences on a real world mesocosm data set.

For count data, we found that the quasi-Poisson model yielded the highest power. The negative binomial GLM resulted in increased Type I errors, which could be fixed using the parametric bootstrap. 
For proportions, binomial GLMs performed better than the linear model, except to determine LOEC at extremely low sample sizes.
The compared non-parametric methods had generally lower power.

We recommend that counts in one-factorial experiments should be analysed using quasi-Poisson models and proportions from counts by binomial GLMs.
These methods should become standard in ecotoxicology.


\section{Introduction}
\label{sec:usetheglm:Introduction}
Ecotoxicologists perform various kinds of experiments yielding different types of data.
Examples are animal counts in mesocosm experiments (non-negative, integer-valued data) or proportions of surviving animals (data bounded between 0 and 1, discrete).
These data are typically not normally distributed. 
Nevertheless, such data are often analysed using methods that assume a normal distribution and variance homogeneity \citep{wang_making_2011}. 
To meet these assumptions data are usually transformed.
For example, ecotoxicological textbooks \citep{newman_quantitative_2012} and guidelines \citep{epa_methods_2002,oecd_current_2006} advise that survival data should be transformed using an arcsine square root transformation. 
For count data from mesocosm experiments a log(Ay + C) transformation is usually applied, where the constants A and C are either chosen arbitrarily or following general recommendations. 
For example, \citet{van_den_brink_impact_2000} suggest to set the term Ay to be 2 for the lowest abundance value (y) greater than zero and C to 1. 
Other transformations, like the square root or fourth root transformation, are also commonly applied in community ecology \cite{anderson_navigating_2011}.
Note that there has been little evaluation and advice for practitioners which transformations to use.
If the transformed data still do not meet the assumptions of the linear model, non-parametric tests are usually applied \cite{wang_making_2011}.

Generalised linear models (GLM) provide a method to analyse counts or proportions from counts in a statistically sound way \cite{nelder_generalized_1972}.
GLMs can handle various types of data distributions, e.g., Poisson or negative binomial (for count data) or binomial (for proportions); the normal distribution being a special case of GLMs.
Despite GLMs being available for more than 40~years, ecotoxicologists do not regularly make use of them.
Recent studies concluded that the linear model should not be applied on transformed data and GLMs be used as they have better statistical properties (\citealt{ohara_not_2010,warton_many_2005} (counts), \citealt{warton_arcsine_2011} (proportions from counts)). 

Ecotoxicological experiments often involve small sample sizes due to practical constraints. 
For example, extremely low samples sizes ($n$~\textless~5) are common in many mesocosm studies \cite{sanderson_pesticide_2002,szocs_analysing_2015}.
Small sample sizes lead to low power in statistical hypothesis testing, on which many ecotoxiological approaches (e.g. risk assessment for pesticides) rely. 
Such an endpoint are L/NOEC values (Lowest / No observed effect concentration).
Although their use has been heavily criticized in the past \cite{laskowski_good_1995}, they are the predominant endpoint in mesocosm experiments \cite{brock_minimum_2015, efsa_ppr_guidance_2013}. 

We explore how GLMs may enhance, when appropriately used, inference in ecotoxicological studies and compared three types of statistical methods (linear model on transformed data, GLM, non-parametric tests).
We first illustrate differences between statistical methods using a data set from a mesocosm study.
Then we further elaborate differences in detecting a general treatment effect and determining the LOEC using simulations of two common data types in ecotoxicology: counts and proportions from counts. 

 \printbibliography

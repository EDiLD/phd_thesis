% -*- root: ../../thesis.tex -*-
\chapter{General Discussion and outlook}
\label{sec:discussion} 
 
\section{Statistical Ecotoxicology}
% NOEC bad
The simulation study performed in chapter~\ref{sec:usetheglm} clearly showed that common experimental designs exhibit unacceptably low statistical power \citep{szocs_statistical_2016, van_der_hoeven_power_1998}.
This underpins the criticism accumulated over the last 30 years towards the usage of NOEC as endpoint \citep{fox_comment_2016}. 
Nevertheless, the NOEC is still one of the standard endpoint for mesocosm experiments in higher tier risk assessment \citep{efsa_guidance_2013}.

% MDD
Recently, \emph{a posteriori} calculations of statistical power have been proposed to counteract these limitations and aid the interpretation treatment-related effects in model ecosystems \citep{brock_minimum_2015}.
The "minimum detectable difference" (MDD) estimates the difference between to means that must exist in order to produce a statistically significant result (p \textless 0.05 \citep{gelman_difference_2006}) and could be used to ínterpret NOEC.
However, \emph{a posteriori} calculations have been shown to have logical flaws when used for interpretation of non-significant results \citep{hoenig_abuse_2001, nakagawa_case_2004}. 
However, conducting and report of \emph{a priori} power calculations, as performed in chapter~\ref{sec:usetheglm}, might provide researchers important information to optimize their study designs, ensuring that their experimental designs have appropriate power \citep{johnson_power_2015}.

% regression design instead of anova designs
Moreover, similar simulations can not only be used to analyse data of factorial designs, but also from regression designs.
Indeed, simulations could be used to determine optimal designs for dose-response models and $EC_x$ determination, balancing precision and resources. 
Regression designs are generally more powerful and provide more information than factorial designs  \citep{cottingham_knowing_2005}. 
Regression designs in mesocosm experiments, assigning the replicates to more tested concentrations, might also provide more insights.
However, currently statistical tools to analyse a community dose-reponse relations, providing a $EC_{x, community}$ are not well explored.
Separate dose-response models could be fit to each species \citep{ritz_toward_2010}, leading to a $EC_x$ for each species in a mesocosm study.
Subsequently, this $EC_x$ values could be combined and summarised using Species Sensitivity Distributions \citep{posthuma_species_2002}, providing a 
hazardous concentration ($HC_{x, community}$) for x \% of species affected \citep{maltby_insecticide_2005}. 
Another possibility would be to use a logistic type of ordination \citep{van_den_brink_multivariate_2003}. 
Reduced-Rank vector generalized linear models (RR-VGLM) could be used to fit such type of models \citep{yee_reduced-rank_2003, yee_vector_2015} but they have not been applied in ecotoxicology yet.

% vgl mit warton, ives und maindonald
In a similar vein, community ecology is currently experiencing a shift towards new class of multivariate methods, incorporating statistical models for abundances across many taxa simultaneously \citep{warton_model-based_2015, warton_distance-based_2012, warton_so_2015, ter_braak_topics_2014}.
However, this methods have not been applied frequently and their applicability to ecotoxicological data is currently unclear \citep{szocs_analysing_2015}. 
All this models have in common, that the choice of statistical model is primarily based ...


% censored data

% discuss also non-detects, for examples (ritz groundwater, bienen,...)
% model based ecology
% UPDATE OECD GUIDELINE
% Warton three things to consider, Maindonald, personal communication.

\section{Leveraging monitoring data for ecological risk assessment}
% Auf verenas paper eingehen
% monitoring anpassen...,

% auf biologische daten eingehen
% monitoring anpassen / synchronisieren


\section{Challenges to utilize 'Big Data' in ecological risk assessment}
% open science -> Letter to the editor of schäfer




\section{Conclusions}


% data will get more important....
% Dealing with censored data?

% ue of reach as an example
% use of censored data => e.g goulsen or ritz

%% Obwohl diss spezieller auf ökotox ausgerichtet, auch nützlich für andere disziplinen (z.b. Münch, twitter,...)
%% 


%% randopm effect multivariate analysis is missing and needed!
%% ----------------------------------------------------------------------------
\section{References}
\printbibliography[heading=none]

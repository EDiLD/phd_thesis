% -*- root: ../thesis.tex -*-

\pdfbookmark[1]{Summary}{Summary}

% \newgeometry{
%   includefoot,
%   right=20mm,
%   left=20mm,
%   bottom=15mm, % space from bottomn to footer (space from text to bottom = footskip + bottom)
%   top=5mm,   }


\thispagestyle{empty}
\begingroup
\let\clearpage\relax
\let\cleardoublepage\relax
\let\cleardoublepage\relax
\chapter*{Summary}
\thispagestyle{empty}
% \setstretch{1}
% \vspace{-5mm}

%Intro
Freshwaters are of immense importance for human livelihood and well-being.
Nevertheless, they are currently facing unprecedented levels of threat from habitat loss and degradation, overexploitation, invasive species and
pollution. To prevent risks from chemical substances, like agricultural pesticides, to the aquatic ecosystem, environmental risk assessment (ERA) is mandatory before using these substances. Concurrently, large-scale environmental monitoring is used for surveillance of the chemical pollution of rivers. 
This thesis examines statistical methods used in ERA, presents a national-scale compilation of chemical monitoring data, an analysis of driver of chemical pollution in streams, a combination with data ERA for a large-scale risk assessment and provides software tools to integrate different datasets used in ERA.

The thesis starts with a brief introduction to ERA, environmental monitoring, and statistical ecotoxicology and gives an overview of the objectives of the thesis.
% use the glm
Chapter~\ref{sec:usetheglm} addresses experimental setups and their statistical analyses using simulations. We show that current designs exhibit unacceptable low statistical power, that statistical methods fitting to the type of data provide higher power and that statistical practices in ERA need to be revised.
% small streams
In chapter~\ref{sec:smallstreams} we compiled all available pesticide monitoring data from Germany.
Hereby, we focused on small streams, similar to those considered in ERA and used threshold concentrations derived during ERA for risk assessment. 
This compilation resulted in the biggest dataset on pesticide exposure currently available for Germany.
Using new statistical techniques, that explicitly take the limits of quantification into account, we demonstrate that 25\% of small streams are exposed to risk from pesticides. 
Especially neonicotinoid pesticides are responsible for this risk.
These risks are associated with agricultural uses and can be detected even at low levels of agricultural use. Moreover, our results indicated that current monitoring underestimates pesticide risks, because of sampling independent of precipitation events.
Additionally, we provide a first large-scale study of annual pesticide exposure dynamics.
% software
Chapters \ref{sec:webchem} and \ref{sec:taxize} describe software solutions to simplify and accelerate the integration of different data sources 
from ERA, environmental monitoring and ecotoxicology.

% conclusion
Overall, this thesis contributes to the emerging discipline of statistical ecotoxicology and shows that pesticides pose a large-scale threat to small streams.
Environmental monitoring can give a post-authorisation feedback to ERA.
However, to protect freshwater ecosystems ERA and environmental monitoring need to be further refined and we provided software solutions to integrate data for this purpose.



\endgroup 
 
% \restoregeometry

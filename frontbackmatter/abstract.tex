% -*- root: ../thesis.tex -*-

\pdfbookmark[1]{Summary}{Summary}

% \newgeometry{
%   includefoot,
%   right=20mm,
%   left=20mm,
%   bottom=15mm, % space from bottomn to footer (space from text to bottom = footskip + bottom)
%   top=5mm,   }


\thispagestyle{empty}
\begingroup
\let\clearpage\relax
\let\cleardoublepage\relax
\let\cleardoublepage\relax
\chapter*{Summary}
\thispagestyle{empty}
% \setstretch{1}
% \vspace{-5mm}

%Intro
Freshwaters are of immense importance for human well-being.
Nevertheless, they are currently facing unprecedented levels of threat from habitat loss and degradation, overexploitation, invasive species and
pollution. 
To prevent risks to aquatic ecosystems chemical substances, like agricultural pesticides, have to pass environmental risk assessment (ERA) before being placed on the market. 
Concurrently, large-scale environmental monitoring is used for surveillance of biological and chemical conditions in freshwaters. 
This thesis examines statistical methods currently used in ERA.
Moreover, it presents a national-scale compilation of chemical monitoring data, an analysis of drivers and dynamics of chemical pollution in streams and, provides a large-scale risk assessment by combination with results from ERA.
Additionally, software tools have been developed to integrate different datasets used in ERA.

The thesis starts with a brief introduction to ERA and environmental monitoring and gives an overview of the objectives of the thesis.
% use the glm
Chapter~\ref{sec:usetheglm} addresses experimental setups and their statistical analyses using simulations. 
The results show that current designs exhibit unacceptably low statistical power, that statistical methods chosen to fit the type of data provide higher power and that statistical practices in ERA need to be revised.
% small streams
In chapter~\ref{sec:smallstreams} we compiled all available pesticide monitoring data from Germany.
Hereby, we focused on small streams, similar to those considered in ERA and used threshold concentrations derived during ERA for a large-scale assessment of threats to freshwaters from pesticides. 
This compilation resulted in the most comprehensive dataset on pesticide exposure currently available for Germany.
Using state-of-the-art statistical techniques, that explicitly take the limits of quantification into account, we demonstrate that 25\% of small streams are at threat from pesticides. 
In particular neonicotinoid pesticides are responsible for these threats.
These are associated with agricultural intensity and can be detected even at low levels of agricultural use. 
Moreover, our results indicated that current monitoring underestimates pesticide risks, because of a sampling decoupled from precipitation events.
Additionally, we provide a first large-scale study of annual pesticide exposure dynamics.
% software
Chapters \ref{sec:webchem} and \ref{sec:taxize} describe software solutions to simplify and accelerate the integration of data from ERA, environmental monitoring and ecotoxicology that is indispensable for the development of landscape-level risk assessment.

% conclusion
Overall, this thesis contributes to the emerging discipline of statistical ecotoxicology and shows that pesticides pose a large-scale threat to small streams.
Environmental monitoring can provide a post-authorisation feedback to ERA.
However, to protect freshwater ecosystems ERA and environmental monitoring need to be further refined and we provide software solutions to utilise existing data for this purpose.



\endgroup 
 
% \restoregeometry
